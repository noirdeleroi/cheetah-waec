

\subsection*{1.8 Ratio and Proportion}



\textbf{Key Concepts:}

\begin{itemize}
    \item \textbf{Ratio:} A ratio compares two or more quantities using division. It can be written as \( a : b \), which means \( \frac{a}{b} \).

    \item \textbf{Simplifying Ratios:} Ratios can be simplified by dividing all terms by their greatest common divisor (GCD).

    \item \textbf{Proportion:} Two quantities are said to be \textbf{proportional} if one quantity is a constant multiple of the other. In direct proportionality, as one quantity increases, the other increases at the same rate, and this relationship can be expressed as \( y \propto x \) or \( y = kx \), where \( k \) is the constant of proportionality. Conversely, two quantities are \textbf{inversely proportional} if one quantity increases while the other decreases such that their product remains constant. This relationship is expressed as \( y \propto \frac{1}{x} \) or \( y = \frac{k}{x} \), where \( k \) is the constant of proportionality.

    \item \textbf{Cross Multiplication:} Used to solve proportions:
    \[
    \frac{a}{b} = \frac{c}{d} \Rightarrow a \cdot d = b \cdot c
    \]

    \item \textbf{Speed, Distance, and Time:} In proportional problems, use the relationship:
    \[
    \text{Speed} = \frac{\text{Distance}}{\text{Time}}
    \]


    \item \textbf{Converting Between Fractions, Decimals, and Percentages:}  
    - Fraction to percentage: multiply by 100  
    - Decimal to percentage: multiply by 100  
    - Percentage to decimal: divide by 100
\end{itemize}





\begin{flushleft}
\textbf{Example 1: Solve the ratio $\frac{2}{5} = \frac{x}{15}$.}

\text{Step 1: Cross multiply the terms in the equation:}

\[
2 \cdot 15 = 5 \cdot x.
\]

\text{Step 2: Simplify the multiplication:}

\[
30 = 5x.
\]

\text{Step 3: Solve for } x:

\[
x = \frac{30}{5} = 6.
\]
Therefore, $x = 6$.
\end{flushleft}

\begin{flushleft}
\textbf{Example 2: Three boys shared D 10,500.00 in the ratio 6:7:8. Find the largest share.}

\text{Step 1: Find the total ratio:}

\[
6 + 7 + 8 = 21.
\]

\text{Step 2: Divide the total amount by the total ratio to find the value of one part:}

\[
\text{Value of one part} = \frac{10,500}{21} = 500.
\]

\text{Step 3: Find the largest share:}

\[
\text{Largest share} = 8 \times 500 = 4,000.
\]
Therefore, the largest share is D 4,000.
\end{flushleft}

\begin{flushleft}
\textbf{Example 3: Given that $P \propto \frac{1}{\sqrt{r}}$ and $P = 3$ when $r = 16$, find the value of $r$ when $P = \frac{3}{2}$.}

Solution:

Step 1: Express $P$ in terms of $r$ using the proportionality constant $k$:
\[
P = \frac{k}{\sqrt{r}}.
\]

Step 2: Substitute $P = 3$ and $r = 16$ to find $k$:
\[
3 = \frac{k}{\sqrt{16}}.
\]
\[
3 = \frac{k}{4}.
\]
\[
k = 3 \times 4 = 12.
\]

Step 3: Use $k = 12$ and $P = \frac{3}{2}$ to find $r$:
\[
\frac{3}{2} = \frac{12}{\sqrt{r}}.
\]
\[
\sqrt{r} = \frac{12}{\frac{3}{2}}.
\]
\[
\sqrt{r} = \frac{12 \times 2}{3} = 8.
\]

Step 4: Square both sides to find $r$:
\[
r = 8^2 = 64.
\]

Therefore, the value of $r$ when $P = \frac{3}{2}$ is $64$.
\end{flushleft}
