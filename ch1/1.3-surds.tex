
\subsection*{1.3 Surds}

Surds are irrational numbers that are expressed using the square root (or other roots) symbol and cannot be simplified to remove the root. They are left in exact form for most mathematical operations.

\textbf{Key Concepts:}
\begin{itemize}
    \item \textbf{Simplifying Surds:}  
    Break the number inside the root into factors, one of which is a perfect square:  
    \[
    \sqrt{50} = \sqrt{25 \times 2} = 5\sqrt{2}
    \]

    \item \textbf{Multiplying and Dividing Surds:}  
    \[
    \sqrt{a} \cdot \sqrt{b} = \sqrt{ab}, \quad \frac{\sqrt{a}}{\sqrt{b}} = \sqrt{\frac{a}{b}}
    \]

    \item \textbf{Rationalizing the Denominator:}  
    Multiply numerator and denominator by a suitable surd to eliminate a root in the denominator:
    \[
    \frac{1}{\sqrt{3}} = \frac{1 \cdot \sqrt{3}}{\sqrt{3} \cdot \sqrt{3}} = \frac{\sqrt{3}}{3}
    \]

    \item \textbf{Using Conjugates:}  
    To rationalize denominators with two terms:
    \[
    \frac{1}{a + \sqrt{b}} = \frac{1 \cdot (a - \sqrt{b})}{(a + \sqrt{b})(a - \sqrt{b})}
    \]
\end{itemize}

\textbf{Examples:}

\begin{flushleft}
\textbf{Example 1: Simplify \(\sqrt{72}\)}

\textbf{Solution:} \vspace{0.2cm}

Step 1: Break into perfect square and other factor:  
\[
\sqrt{72} = \sqrt{36 \times 2} = \sqrt{36} \cdot \sqrt{2} = 6\sqrt{2}
\]
\end{flushleft}

\begin{flushleft}
\textbf{Example 2: Rationalize \(\frac{5}{\sqrt{2}}\)}

\textbf{Solution:} \vspace{0.2cm}

Step 1: Multiply numerator and denominator by \(\sqrt{2}\):  
\[
\frac{5}{\sqrt{2}} \cdot \frac{\sqrt{2}}{\sqrt{2}} = \frac{5\sqrt{2}}{2}
\]
\end{flushleft}

\begin{flushleft}
\textbf{Example 3: Rationalize \(\frac{3}{2+\sqrt{3}}\)}

\textbf{Solution:} \vspace{0.2cm}

Step 1: Multiply numerator and denominator by the conjugate \(2 - \sqrt{3}\):  
\[
\frac{3}{2+\sqrt{3}} \cdot \frac{2 - \sqrt{3}}{2 - \sqrt{3}} = \frac{3(2 - \sqrt{3})}{(2+\sqrt{3})(2-\sqrt{3})}
\]

Step 2: Simplify:  
\[
\text{Numerator: } 6 - 3\sqrt{3} 
\]
\text{Denominator: } 4 - 3 = 1
\[
\]

Final answer:  
\[
6 - 3\sqrt{3}
\]
\end{flushleft}

