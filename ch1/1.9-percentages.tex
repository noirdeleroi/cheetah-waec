
\subsection*{1.9 Percentages}

Percentages are a way of expressing a number as a fraction of 100. The symbol for percent is "\%". Percentages are widely used in comparing ratios, increases and decreases, discounts, profits, and more.

\textbf{Key Concepts:}
\begin{itemize}
    \item \textbf{To convert a fraction or decimal to a percentage:} Multiply by 100.
    \[
    \frac{3}{4} = 0.75 = 75\%
    \]
    
    \item \textbf{To convert a percentage to a decimal:} Divide by 100.
    \[
    40\% = \frac{40}{100} = 0.4
    \]

    \item \textbf{Percentage Increase/Decrease:}
    \[
    \text{Percentage Change} = \frac{\text{Change}}{\text{Original Value}} \times 100\%
    \]

    \item \textbf{Finding a percentage of a number:}
    \[
    \text{e.g., } 25\% \text{ of } 200 = \frac{25}{100} \times 200 = 50
    \]
\end{itemize}
\textbf{Examples:}

\begin{flushleft}
\textbf{Example 1: Converting a Fraction to a Percentage}

Convert \( \frac{5}{8} \) to a percentage.

\textbf{Solution:} \vspace{0.2cm}

\[
\frac{5}{8} \times 100 = 62.5\%
\]
\end{flushleft}

\begin{flushleft}
\textbf{Example 2: Finding Percentage of a Quantity}

Find 30\% of 250.

\textbf{Solution:} \vspace{0.2cm}

\[
\frac{30}{100} \times 250 = 75
\]
\end{flushleft}

\begin{flushleft}
\textbf{Example 3: Calculating Percentage Increase}

A quantity increases from 40 to 50. Find the percentage increase.

\textbf{Solution:} \vspace{0.2cm}

\[
\text{Increase} = 50 - 40 = 10, \quad \text{Percentage Increase} = \frac{10}{40} \times 100 = 25\%
\]
\end{flushleft}

\begin{flushleft}
\textbf{Example 4: Calculating Percentage Decrease}

A value drops from 80 to 60. Find the percentage decrease.

\textbf{Solution:} \vspace{0.2cm}

\[
\text{Decrease} = 80 - 60 = 20, \quad \text{Percentage Decrease} = \frac{20}{80} \times 100 = 25\%
\]
\end{flushleft}

