
\subsection*{1.6 Standard Form of a Number}

Standard form (also known as scientific notation) is a way of expressing very large or very small numbers in the form:
\[
a \times 10^n \quad \text{where } 1 \leq a < 10 \text{ and } n \text{ is an integer}
\]

\textbf{Key Concepts:}
\begin{itemize}
    \item For large numbers, \(n\) is positive and indicates how many places to move the decimal to the right.
    \item For small numbers, \(n\) is negative and indicates how many places to move the decimal to the left.
\end{itemize}

\textbf{Examples:}

\begin{flushleft}
\textbf{Example 1: Converting a large number to standard form}

Write \( 4500000 \) in standard form.

\textbf{Solution:} \vspace{0.2cm}

Step 1: Move the decimal 6 places to the left:  
\[
4500000 = 4.5 \times 10^6
\]
\end{flushleft}

\begin{flushleft}
\textbf{Example 2: Converting a small number to standard form}

Write \( 0.00032 \) in standard form.

\textbf{Solution:} \vspace{0.2cm}

Step 1: Move the decimal 4 places to the right:  
\[
0.00032 = 3.2 \times 10^{-4}
\]
\end{flushleft}

\begin{flushleft}
\textbf{Example 3: Multiplying numbers in standard form}

Simplify:  
\[
(2.7 \times 10^{-4}) \times (6.3 \times 10^6)
\]

\textbf{Solution:} \vspace{0.2cm}

Step 1: Multiply the decimal parts:  
\[
2.7 \times 6.3 = 17.01
\]

Step 2: Multiply the powers of ten:  
\[
10^{-4} \times 10^6 = 10^2
\]

Step 3: Combine and write in standard form:  
\[
17.01 \times 10^2 = 1.701 \times 10^3
\]
\end{flushleft}

