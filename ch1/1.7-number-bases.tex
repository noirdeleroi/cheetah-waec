\subsection*{1.7 Number Bases}

Number bases represent the numerical system in which values are expressed. The base indicates the number of unique digits, including zero, that a numeral system uses. Common bases include:

\begin{itemize}
    \item \textbf{Base 10 (Decimal):} Standard numeric system.
    \item \textbf{Base 2 (Binary):} Used in computer systems.
    \item \textbf{Base 16 (Hexadecimal):} Common in programming and computing.
\end{itemize}

\textbf{Key Concepts:}
\begin{itemize}
    \item Converting numbers between bases involves successive division by the new base and noting remainders.
    \item To perform operations on numbers in any base, first convert them to base 10, perform the operation, and then convert the answer back to the original base.
\end{itemize}

\begin{flushleft}
\textbf{Example 1: Conversion from Decimal to Binary}

Convert \(18\) from base 10 to base 2.

\textbf{Solution:} \vspace{0.2cm}

Step 1: Divide by 2 and record remainders:  
\[
18 \div 2 = 9 \text{ remainder } 0
\]
\[
9 \div 2 = 4 \text{ remainder } 1
\]
\[
4 \div 2 = 2 \text{ remainder } 0
\]
\[
2 \div 2 = 1 \text{ remainder } 0
\]
\[
1 \div 2 = 0 \text{ remainder } 1
\]

Step 2: Write the remainders in reverse order:  
\[
18_{10} = 10010_2
\]
\end{flushleft}

\begin{flushleft}
\textbf{Example 2: Conversion from Binary to Decimal}

Convert \(1011_2\) to base 10.

\textbf{Solution:} \vspace{0.2cm}

Apply place values:  
\[
(1 \cdot 2^3) + (0 \cdot 2^2) + (1 \cdot 2^1) + (1 \cdot 2^0) = 8 + 0 + 2 + 1 = 11
\]

Therefore, \(1011_2 = 11_{10}\).
\end{flushleft}

\begin{flushleft}
\textbf{Example 3: Multiplication in Base 3}

Simplify \(21_3 \cdot 2_3\) in base 3.

\textbf{Solution:} \vspace{0.2cm}

Convert each to decimal:  
\[
21_3 = (2 \cdot 3^1) + (1 \cdot 3^0) = 6 + 1 = 7, \quad 2_3 = 2
\]

Multiply in decimal:  
\[
7 \cdot 2 = 14
\]

Convert 14 to base 3:  
\[
14 \div 3 = 4 \text{ remainder } 2, \quad 4 \div 3 = 1 \text{ remainder } 1, \quad 1 \div 3 = 0 \text{ remainder } 1
\]

Write the remainders in reverse:  
\[
14_{10} = 112_3
\]
\end{flushleft}

\begin{flushleft}
\textbf{Example 4: Squaring in Base 2}

Simplify \((11_{\text{two}})^2\).

\textbf{Solution:} \vspace{0.2cm}

Convert to decimal:  
\[
11_{\text{two}} = 3
\]

Square in decimal:  
\[
3^2 = 9
\]

Convert 9 back to binary:  
\[
9 \div 2 = 4 \text{ remainder } 1
\]
\[
4 \div 2 = 2 \text{ remainder } 0
\]
\[
2 \div 2 = 1 \text{ remainder } 0
\]
\[
1 \div 2 = 0 \text{ remainder } 1
\]

Write in reverse order:  
\[
9_{10} = 1001_2
\]
\end{flushleft}
