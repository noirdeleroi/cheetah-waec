
\subsection*{1.1 Fractions and Decimals}
Fractions and decimals are fundamental concepts in arithmetic. Operations with fractions include addition, subtraction, multiplication, and division. Fractions can also be simplified to their lowest terms by determining the greatest common factor (GCF) of the numerator and denominator.

Decimals are numbers written in a base-10 system and can be converted to fractions and vice versa. 

\textbf{Key Concepts:}
\begin{itemize}
    \item \textbf{Least Common Multiple (LCM)}: The smallest number that is a multiple of two or more numbers.
    \item \textbf{Highest Common Factor (HCF)}: The largest number that divides two or more numbers without leaving a remainder.
    \item \textbf{Factorization}: Breaking a number into its prime factors.
    \item \textbf{Addition and Subtraction of Fractions:}  
To add or subtract fractions, first find a common denominator (usually the least common multiple of the denominators), then combine the numerators.  
\[
\frac{a}{b} + \frac{c}{d} = \frac{ad + bc}{bd}, \quad
\frac{a}{b} - \frac{c}{d} = \frac{ad - bc}{bd}
\]

\item \textbf{Multiplication of Fractions:}  
Multiply the numerators together and the denominators together:
\[
\frac{a}{b} \times \frac{c}{d} = \frac{ac}{bd}
\]

\item \textbf{Division of Fractions:}  
Multiply the first fraction by the reciprocal of the second:
\[
\frac{a}{b} \div \frac{c}{d} = \frac{a}{b} \times \frac{d}{c} = \frac{ad}{bc}
\]

\item \textbf{Mixed Numbers:}  
A mixed number consists of a whole number and a proper fraction. To perform operations, convert mixed numbers to improper fractions:
\[
a\frac{b}{c} = \frac{ac + b}{c}
\]
Then proceed with the appropriate operation.

\end{itemize}



\begin{flushleft}
\textbf{Example 1: Addition of Fractions}

Simplify:  
\[
\frac{3}{4} + \frac{5}{6}
\]

\textbf{Solution:} \vspace{0.2cm}

Step 1: Find the LCM of the denominators 4 and 6:  
\[
\text{LCM} = 12
\]

Step 2: Convert each fraction to have a denominator of 12:  
\[
\frac{3}{4} = \frac{9}{12}, \quad \frac{5}{6} = \frac{10}{12}
\]

Step 3: Add the fractions:  
\[
\frac{9}{12} + \frac{10}{12} = \frac{19}{12}
\]

Therefore, the result is:  
\[
\frac{19}{12}
\]
\end{flushleft}


\begin{flushleft}
\textbf{Example 2: Multiplication of Fractions}

Simplify:  
\[
\frac{2}{3} \times \frac{4}{5}
\]

\textbf{Solution:}

Multiply the numerators and the denominators:
\[
\frac{2 \times 4}{3 \times 5} = \frac{8}{15}
\]

Therefore, the result is:
\[
\frac{8}{15}
\]
\end{flushleft}

\begin{flushleft}
\textbf{Example 3: Division of Fractions}

Simplify:  
\[
\frac{5}{6} \div \frac{2}{9}
\]

\textbf{Solution:}

To divide, multiply by the reciprocal:
\[
\frac{5}{6} \div \frac{2}{9} = \frac{5}{6} \times \frac{9}{2} = \frac{5 \times 9}{6 \times 2} = \frac{45}{12}
\]

Simplify the result:
\[
\frac{45}{12} = \frac{15}{4}
\]

Therefore, the answer is:
\[
\frac{15}{4}
\]
\end{flushleft}

\begin{flushleft}
\textbf{Example 4: Multiplication of Decimals}

Simplify:  
\[
1.25 \times 0.4
\]

\textbf{Solution:} \vspace{0.2cm}

Step 1: Multiply as if they are whole numbers:  
\[
125 \times 4 = 500
\]

Step 2: Count the total number of decimal places:  
1.25 has 2 decimal places, and 0.4 has 1 decimal place, so total = 3.

Step 3: Adjust the product by placing the decimal point:  
\[
500 \rightarrow 0.500
\]

Therefore, the result is:  
\[
0.5
\]
\end{flushleft}

\begin{flushleft}
\textbf{Example 5: Factorization of a Number}

Factorize 84 into its prime factors.

\textbf{Solution:} \vspace{0.2cm}

Step 1: Divide successively by prime numbers:  
\[
84 \div 2 = 42, \quad 42 \div 2 = 21, \quad 21 \div 3 = 7
\]

Step 2: Write the product of prime factors:  
\[
84 = 2^2 \times 3 \times 7
\]
\end{flushleft}

\begin{flushleft}
\textbf{Example 6: Finding the HCF}

Find the HCF of 48 and 180.

\textbf{Solution:} \vspace{0.2cm}

Step 1: Perform prime factorization:  
\[
48 = 2^4 \times 3, \quad 180 = 2^2 \times 3^2 \times 5
\]

Step 2: Identify common prime factors:  
Common factors are \(2^2\) and \(3\)

Step 3: Multiply the common factors:  
\[
\text{HCF} = 2^2 \times 3 = 12
\]
\end{flushleft}

\begin{flushleft}
\textbf{Example 7: Simplify an Expression with Mixed Numbers}

Simplify:  
\[
\frac{1\frac{7}{8} \times 2\frac{2}{5}}{6\frac{3}{4} \div \frac{3}{4}}
\]

\textbf{Solution:} \vspace{0.2cm}

Step 1: Convert all mixed numbers to improper fractions:
\[
1\frac{7}{8} = \frac{15}{8}, \quad 2\frac{2}{5} = \frac{12}{5}, \quad 6\frac{3}{4} = \frac{27}{4}
\]

Step 2: Multiply the fractions in the numerator:
\[
\frac{15}{8} \times \frac{12}{5} = \frac{180}{40} = \frac{9}{2}
\]

Step 3: Divide the fractions in the denominator:
\[
\frac{27}{4} \div \frac{3}{4} = \frac{27}{4} \times \frac{4}{3} = \frac{108}{12} = 9
\]

Step 4: Divide the numerator by the denominator:
\[
\frac{\frac{9}{2}}{9} = \frac{9}{2} \times \frac{1}{9} = \frac{1}{2}
\]

Final answer:
\[
\boxed{\frac{1}{2}}
\]
\end{flushleft}

