
\subsection*{1.4 Exponential Equations}

Exponential equations are equations in which the variable appears in the exponent. Solving them involves applying the laws of indices and using techniques like expressing both sides with the same base or applying logarithms.

\textbf{Key Concepts:}
\begin{itemize}
    \item If \( a^x = a^y \), then \( x = y \) (provided \( a \neq 1 \))
    \item Use the laws of indices (Chapter 1.2) to simplify exponential expressions
    \item If the bases are not the same, convert them to a common base if possible
    \item If not possible, use logarithms to solve
\end{itemize}

\begin{flushleft}
\textbf{Example 1: Same Base}

Solve:  
\[
2^x = 16
\]

\textbf{Solution:} \vspace{0.2cm}

Step 1: Express 16 as a power of 2:  
\[
16 = 2^4
\]

Step 2: Since the bases are the same:  
\[
2^x = 2^4 \Rightarrow x = 4
\]
\end{flushleft}

\begin{flushleft}
\textbf{Example 2: Using Logarithms}

Solve:  
\[
3^x = 20
\]

\textbf{Solution:} \vspace{0.2cm}

Step 1: Take logarithm of both sides:  
\[
\log(3^x) = \log(20)
\]

Step 2: Use the power rule of logarithms:  
\[
x \log 3 = \log 20
\]

Step 3: Solve for \(x\):  
\[
x = \frac{\log 20}{\log 3} \approx \frac{1.3010}{0.4771} \approx 2.73
\]
\end{flushleft}
