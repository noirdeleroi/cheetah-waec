
\subsection*{2.8 Sequences and Series}
A sequence is an ordered list of numbers that follow a specific pattern, and a series is the sum of the terms of a sequence.

\textbf{Key Concepts:}
\begin{itemize}
    \item \textbf{Sequence Definitions:}
    \begin{itemize}
        \item \textbf{Sequence:} An ordered list of numbers arranged according to a specific rule or pattern. Each number in the sequence is called a term, and the position of a term is denoted by $n$. Sequences can be defined by a general formula (explicit rule) or a recursive formula. For example: 5, 10, 15, 20,... is a sequence with first term $a_1 = 5$, second term $a_2 = 10$ and so on.

        \item \textbf{General Formula:} A formula that defines the $n$th term of a sequence, $a_n$, in terms of $n$. For examplme $a_n=n^2-2$ will give a sequence -1, 2, 7, 14, 23,...
        \item \textbf{Recursive Formula:} A formula that defines each term of a sequence in relation to one or more previous terms. For example, $a_1 = 2$, $a_{n+1} = a_n + 3$ will give a sequence 2, 5, 8, 11,...
    \end{itemize}
    \item \textbf{Sequence Evaluation:}  
To evaluate a sequence means to find specific terms using the general formula (nth term) or recursive rule. 

\textbf{Example:}  
Find the 5th term of the sequence defined by \(T_n = 3n - 2\):  
\[
T_5 = 3(5) - 2 = 15 - 2 = 13
\]

    \item \textbf{Arithmetic Progression (AP):} A sequence where each term increases or decreases by a constant value, called the common difference $d$. The general form is:
    \[
    a, a+d, a+2d, \dots
    \]
        \begin{itemize}
            \item \textbf{Nth Term of AP:}
            \[
            a_n = a + (n-1)d
            \]
            where $a$ is the first term, $d$ is the common difference, and $n$ is the term number.
            \item \textbf{Sum of the First $n$ Terms of AP:}
            \[
            S_n = \frac{n}{2} [2a + (n-1)d]
            \]
        \end{itemize}
    \item \textbf{Geometric Progression (GP):} A sequence where each term is multiplied by a constant value, called the common ratio $r$. The general form is:
    \[
    a, ar, ar^2, \dots
    \]
        \begin{itemize}
            \item \textbf{Nth Term of GP:}
            \[
            a_n = ar^{n-1}
            \]
            where $a$ is the first term, $r$ is the common ratio, and $n$ is the term number.
            \item \textbf{Sum of the First $n$ Terms of GP:}
            \[
            S_n = a \frac{1-r^n}{1-r}, \quad r \neq 1
            \]
            \item \textbf{Sum to Infinity of GP:}
            \[
            S_\infty = \frac{a}{1-r}, \quad |r| < 1
            \]
        \end{itemize}
    \item \textbf{Sum of Sequences:} The sum of the terms in a sequence, calculated using the formulas above depending on whether the sequence is arithmetic or geometric.
\end{itemize}

\begin{flushleft}
\textbf{Example 1: Evaluating Terms in a Sequence}

If \( U_n = n(n^2 + 1) \), evaluate \( U_5 - U_4 \).

\textbf{Solution:} \vspace{0.2cm}

Step 1: Find \( U_5 \):
\[
U_5 = 5(5^2 + 1) = 5(25 + 1) = 5 \times 26 = 130
\]

Step 2: Find \( U_4 \):
\[
U_4 = 4(4^2 + 1) = 4(16 + 1) = 4 \times 17 = 68
\]

Step 3: Subtract:
\[
U_5 - U_4 = 130 - 68 = \boxed{62}
\]
\end{flushleft}


\begin{flushleft}
\textbf{Example 2: Find the 10th term of the AP $3, 7, 11, \dots$.}

\vspace{0.5cm}
\textbf{Solution:}
\vspace{0.5cm}

Identify the values:
\[
a = 3, \quad d = 7 - 3 = 4, \quad n = 10.
\]

Use the formula for the $n$th term:
\[
a_n = a + (n-1)d = 3 + (10-1) \cdot 4 = 3 + 36 = 39.
\]

Therefore, the 10th term is 39.
\end{flushleft}

\begin{flushleft}
\textbf{Example 3: Find the sum of the first 8 terms of the AP $5, 9, 13, \dots$.}

\vspace{0.5cm}
\textbf{Solution:}
\vspace{0.5cm}

Identify the values:
\[
a = 5, \quad d = 9 - 5 = 4, \quad n = 8.
\]

Use the formula for the sum of the first $n$ terms:
\[
S_n = \frac{n}{2} [2a + (n-1)d] = \frac{8}{2} [2 \cdot 5 + (8-1) \cdot 4].
\]

Simplify:
\[
S_n = 4 [10 + 28] = 4 \cdot 38 = 152.
\]

Therefore, the sum of the first 8 terms is 152.
\end{flushleft}

\begin{flushleft}
\textbf{Example 4: Find the 6th term of the GP $2, 4, 8, \dots$.}

\vspace{0.5cm}
\textbf{Solution:}
\vspace{0.5cm}

Identify the values:
\[
a = 2, \quad r = 4 \div 2 = 2, \quad n = 6.
\]

Use the formula for the $n$th term:
\[
a_n = ar^{n-1} = 2 \cdot 2^{6-1} = 2 \cdot 2^5 = 2 \cdot 32 = 64.
\]

Therefore, the 6th term is 64.
\end{flushleft}

\begin{flushleft}
\textbf{Example 5: Find the sum to infinity of the GP $3, 1.5, 0.75, \dots$.}

\vspace{0.5cm}
\textbf{Solution:}
\vspace{0.5cm}

Identify the values:
\[
a = 3, \quad r = 1.5 \div 3 = 0.5.
\]

Use the formula for the sum to infinity:
\[
S_\infty = \frac{a}{1-r} = \frac{3}{1-0.5} = \frac{3}{0.5} = 6.
\]

Therefore, the sum to infinity is 6.
\end{flushleft}
