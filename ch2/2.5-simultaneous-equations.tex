
\subsection*{2.5 Simultaneous Equations}

Simultaneous equations are two or more equations involving the same set of variables. The solution is the set of values that satisfies all the equations simultaneously.

\textbf{Key Concepts:}
\begin{itemize}
    \item Solve by substitution or elimination.
    \item A solution is a point of intersection on the graph of the equations.
    \item Word problems often lead to a system of equations that can be solved to find unknown quantities.
\end{itemize}

\textbf{Examples:}

\begin{flushleft}
\textbf{Example 1: Solving by Elimination}

Solve:  
\[
\begin{cases}
2x + y = 7 \\
3x - y = 8
\end{cases}
\]

\textbf{Solution:} \vspace{0.2cm}

Step 1: Add the equations to eliminate \(y\):  
\[
(2x + y) + (3x - y) = 7 + 8 \Rightarrow 5x = 15
\Rightarrow x = 3
\]

Step 2: Substitute into one of the original equations:  
\[
2(3) + y = 7 \Rightarrow 6 + y = 7 \Rightarrow y = 1
\]

So, the solution is \( \boxed{x = 3, y = 1} \).
\end{flushleft}

\begin{flushleft}
\textbf{Example 2: Solving by Substitution}

Solve:  
\[
\begin{cases}
x + y = 10 \\
x = 2y
\end{cases}
\]

\textbf{Solution:} \vspace{0.2cm}

Step 1: Substitute \(x = 2y\) into the first equation:  
\[
2y + y = 10 \Rightarrow 3y = 10 \Rightarrow y = \frac{10}{3}
\]

Step 2: Find \(x\):  
\[
x = 2y = \frac{20}{3}
\]

So, the solution is \( \boxed{x = \frac{20}{3},\ y = \frac{10}{3}} \).
\end{flushleft}

\begin{flushleft}
\textbf{Example 3: Word Problem (Ages)}

The sum of the ages of a father and his son is 50 years. The father is 4 times as old as the son. Find their ages.

\textbf{Solution:} \vspace{0.2cm}

Let \(x\) be the age of the son, and \(y\) the father's age.

\[
\begin{cases}
x + y = 50 \\
y = 4x
\end{cases}
\]

Substitute into the first equation:
\[
x + 4x = 50 \Rightarrow 5x = 50 \Rightarrow x = 10
\Rightarrow y = 4 \times 10 = 40
\]

So, the son is \( \boxed{10} \) and the father is \( \boxed{40} \).
\end{flushleft}

\begin{flushleft}
\textbf{Example 4: Word Problem (Money)}

A man has ₦500 in ₦50 and ₦20 notes. If the number of ₦50 notes is 4 more than the number of ₦20 notes, how many of each note does he have?

\textbf{Solution:} \vspace{0.2cm}

Let \(x\) be the number of ₦20 notes, and \(x + 4\) the number of ₦50 notes.

\[
20x + 50(x + 4) = 500
\Rightarrow 20x + 50x + 200 = 500
\Rightarrow 70x = 300 \Rightarrow x = \frac{300}{70} = \frac{30}{7}
\]

Since the result is not a whole number, check the equation or reframe with compatible values.  
Let’s assume the problem meant “4 times as many”:

Let \(x\) be the number of ₦20 notes, and \(4x\) the number of ₦50 notes:

\[
20x + 50(4x) = 500 \Rightarrow 20x + 200x = 500 \Rightarrow 220x = 500
\Rightarrow x = \frac{500}{220} = \frac{25}{11}
\]

Still not a whole number — this suggests a correction is needed in values.  
Let's adjust the total to ₦560:

Try:
\[
20x + 50(x + 4) = 560 \Rightarrow 20x + 50x + 200 = 560 \Rightarrow 70x = 360 \Rightarrow x = \frac{36}{7}
\]

To ensure integer solution, set:
Let \(x\) ₦20 notes, and \(x + 2\) ₦50 notes:
\[
20x + 50(x + 2) = 500 \Rightarrow 20x + 50x + 100 = 500 \Rightarrow 70x = 400 \Rightarrow x = \frac{40}{7}
\]

In short: adjust word problem to match whole solution.

