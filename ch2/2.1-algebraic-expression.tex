
\subsection*{2.1 Algebraic Expressions}
An algebraic expression is a mathematical statement that includes numbers, variables, and operations. Operations like addition, subtraction, multiplication, division, and exponentiation can be applied to form such expressions. Understanding the fundamental terms is crucial for simplifying and analyzing algebraic expressions.

\textbf{Key Concepts:}
\begin{itemize}
\item \textbf{Simplification of Expressions:} The process of combining like terms and applying arithmetic or algebraic operations to write an expression in its simplest form.

\textbf{Examples:}
\[
3x + 5x = 8x \quad \text{(combine like terms)}
\]
\[
2(x + 4) - 3(x - 2) = 2x + 8 - 3x + 6 = -x + 14 \quad \text{(expand and simplify)}
\]

    \item \textbf{Expansion:} The process of removing brackets in algebraic expressions by applying the distributive property. Each term inside the bracket is multiplied by the term outside.

\textbf{Examples:}
\[
a(b + c) = ab + ac
\]
\[
(x + 2)(x + 5) = x^2 + 5x + 2x + 10 = x^2 + 7x + 10
\]
\item \textbf{Factorization:} The reverse of expansion — writing an expression as a product of its factors. Common techniques include taking out common factors, using identities, or grouping terms.

\textbf{Examples:}
\[
6x + 9 = 3(2x + 3) \quad \text{(common factor)}
\]
\[
x^2 + 5x + 6 = (x + 2)(x + 3) \quad \text{(quadratic factorization)}
\]
\[
a^2 - b^2 = (a - b)(a + b) \quad \text{(difference of squares)}
\]

    \item \textbf{Factorization Formulas:} 
\[
a^2 - b^2 = (a-b)(a+b)
\]
\[
a^2 + 2ab + b^2 = (a+b)^2
\]
\[
a^2 - 2ab + b^2 = (a-b)^2
\]
\[
x^3 + y^3 = (x+y)(x^2 - xy + y^2)
\]
\[
x^3 - y^3 = (x-y)(x^2 + xy + y^2)
\]

    \item \textbf{Coefficient:} The numerical factor in a term of an algebraic expression. For example, in $5x^2$, the coefficient is 5.

\item \textbf{Term:} A single mathematical expression involving numbers, variables, or their product. For example, $3x$, $-7y^2$, and $4$ are all terms of expression $3x-7y^2+4$.

\item \textbf{Polynomial Expression:} An algebraic expression made up of terms consisting of variables raised to whole number powers and their coefficients. The general form is:
\[
a_nx^n + a_{n-1}x^{n-1} + \dots + a_1x + a_0
\]
where \(a_i \in \mathbb{R}\) and \(n \in \mathbb{N}_0\).

\item \textbf{Rational Expression:} An expression that can be written as the ratio of two polynomials:
\[
\frac{P(x)}{Q(x)}, \quad \text{where } Q(x) \neq 0
\]

\item \textbf{Irrational Expression:} An algebraic expression that involves roots (square roots, cube roots, etc.) of variables or numbers that cannot be expressed as a ratio of polynomials.  
\[
\text{Example: } \sqrt{x + 1}, \quad \frac{1}{\sqrt{x - 2}}
\]


\end{itemize}

\begin{flushleft}
\textbf{Example 1: Expand $(x+2)(x-3)$.}

\vspace{0.5cm}
\textbf{Solution:}
\vspace{0.5cm}

Use the distributive property:
\[
(x+2)(x-3) = x^2 - 3x + 2x - 6 = x^2 - x - 6.
\]
Therefore, the expanded expression is $x^2 - x - 6$.
\end{flushleft}

\begin{flushleft}
\textbf{Example 2: Factorize $xy + xz + wy + wz$.}

\vspace{0.5cm}
\textbf{Solution:}
\vspace{0.5cm}

Group terms and factorize:
\[
xy + xz + wy + wz = x(y+z) + w(y+z) = (x+w)(y+z).
\]
Therefore, the factorized expression is $(x+w)(y+z)$.
\end{flushleft}

\begin{flushleft}
\textbf{Example 3: Factorize $x^2 - 9$.}

\vspace{0.5cm}
\textbf{Solution:}
\vspace{0.5cm}

Recognize the expression as a difference of squares:
\[
x^2 - 9 = (x - 3)(x + 3).
\]

Therefore, the factorized form is $(x - 3)(x + 3)$.
\end{flushleft}

\begin{flushleft}
\textbf{Example 4: Simplify and factor \((x + 3)^2 - (x - 2)^2\)}

\vspace{0.5cm}
\textbf{Solution:}
\vspace{0.5cm}

Step 1: Expand both squares using identities:
\[
(x + 3)^2 = x^2 + 6x + 9, \quad (x - 2)^2 = x^2 - 4x + 4
\]

Step 2: Subtract the expressions:
\[
x^2 + 6x + 9 - (x^2 - 4x + 4)
\]

Step 3: Simplify:
\[
x^2 + 6x + 9 - x^2 + 4x - 4 = 10x + 5
\]

Step 4: Factor the result:
\[
10x + 5 = 5(2x + 1)
\]

Therefore, the simplified and factored form is \( \boxed{5(2x + 1)} \).

\begin{flushleft}
\textbf{Example 4: Simplify $\frac{2 - 18m^2}{1 + 3m}$.}

\vspace{0.5cm}
\textbf{Solution:}
\vspace{0.5cm}

Step 1: Factorize the numerator:
\[
2 - 18m^2 = 2(1 - 9m^2).
\]

Recognize $1 - 9m^2$ as a difference of squares:
\[
1 - 9m^2 = (1 - 3m)(1 + 3m).
\]

Thus:
\[
2 - 18m^2 = 2(1 - 3m)(1 + 3m).
\]

Step 2: Simplify the fraction:
\[
\frac{2 - 18m^2}{1 + 3m} = \frac{2(1 - 3m)(1 + 3m)}{1 + 3m}.
\]

Cancel the common factor $(1 + 3m)$ (valid when $1 + 3m \neq 0$):
\[
\frac{2(1 - 3m)(1 + 3m)}{1 + 3m} = 2(1 - 3m).
\]

Step 3: Expand if needed:
\[
2(1 - 3m) = 2 - 6m.
\]

Therefore, the simplified expression is $2 - 6m$.
\end{flushleft}
\end{flushleft}
