
\subsection*{2.3 Domain and Range}

The \textbf{domain} of a function is the set of all possible input values (usually \(x\)) for which the function is defined.  
The \textbf{range} is the set of all possible output values (usually \(y\)) the function can produce.

\textbf{Key Concepts:}
\begin{itemize}
    \item \textbf{Rational Expressions:} Denominators must not be zero. Exclude values of \(x\) that make the denominator zero.
    \[
    f(x) = \frac{1}{x - 3} \quad \text{Domain: } x \neq 3
    \]

    \item \textbf{Irrational Expressions:} Even roots must have non-negative radicands (the expression inside the root).
    \[
    f(x) = \sqrt{x - 2} \quad \text{Domain: } x \geq 2
    \]

    \item \textbf{Logarithmic Functions:} The argument of the logarithm must be positive.
    \[
    f(x) = \log(x - 1) \quad \text{Domain: } x > 1
    \]
    \item \textbf{Undefined Expressions:} These occur when mathematical operations are not valid for certain input values. Common cases include:
\begin{itemize}
    \item Division by zero:
    \[
    \frac{1}{x - 2} \quad \text{is undefined at } x = 2
    \]
    \item Even roots of negative numbers (in real numbers):
    \[
    \sqrt{x - 5} \quad \text{is undefined for } x < 5
    \]
    \item Logarithm of non-positive numbers:
    \[
    \log(x) \quad \text{is undefined for } x \leq 0
    \]
\end{itemize}
To find the domain, exclude values of \(x\) that make the expression undefined.

\end{itemize}

\textbf{Examples:}

\begin{flushleft}
\textbf{Example 1: Domain of a Rational Function}

Find the domain of \( f(x) = \frac{2x + 1}{x^2 - 4} \).

\textbf{Solution:} \vspace{0.2cm}

Step 1: Set the denominator not equal to zero:
\[
x^2 - 4 \neq 0 \Rightarrow x \neq \pm2
\]

Therefore, the domain is all real numbers except \( x = -2 \) and \( x = 2 \).
\end{flushleft}

\begin{flushleft}
\textbf{Example 2: Domain of a Root Function}

Find the domain of \( f(x) = \sqrt{5 - x} \).

\textbf{Solution:} \vspace{0.2cm}

The expression under the square root must be \(\geq 0\):

\[
5 - x \geq 0 \Rightarrow x \leq 5
\]

So, the domain is \( x \leq 5 \).
\end{flushleft}

\begin{flushleft}
\textbf{Example 3: Domain of a Logarithmic Function}

Find the domain of \( f(x) = \log(x^2 - 9) \).

\textbf{Solution:} \vspace{0.2cm}

The argument of the log must be positive:
\[
x^2 - 9 > 0 \Rightarrow (x - 3)(x + 3) > 0
\]

This inequality is satisfied when \( x < -3 \) or \( x > 3 \).  
So the domain is \( (-\infty, -3) \cup (3, \infty) \).
\end{flushleft}
\begin{flushleft}
\textbf{Example 4: Undefined Value in a Rational Expression}

Find the value of \(x\) for which the expression  
\[
f(x) = \frac{x + 5}{x + 7}
\]  
is undefined.

\textbf{Solution:} \vspace{0.2cm}

A rational expression is undefined when the denominator is zero.

Set the denominator equal to zero:  
\[
x + 7 = 0 \Rightarrow x = -7
\]

Therefore, the expression is undefined at \( \boxed{x = -7} \).
\end{flushleft}

\begin{flushleft}
\textbf{Example 5: Undefined Values in a Rational Expression}

Find the values of \(x\) for which the expression  
\[
f(x) = \frac{1}{(x - 3)(x^2 - 9)}
\]  
is undefined.

\textbf{Solution:} \vspace{0.2cm}

Step 1: Set the denominator equal to zero:
\[
(x - 3)(x^2 - 9) = 0
\]

Step 2: Factor further:
\[
x^2 - 9 = (x - 3)(x + 3)
\]

So the full denominator becomes:
\[
(x - 3)(x - 3)(x + 3)
\]

Step 3: Set each factor equal to zero:
\[
x - 3 = 0 \Rightarrow x = 3
\]
\[
x + 3 = 0 \Rightarrow x = -3
\]

Therefore, the expression is undefined at \( \boxed{x = -3 \text{ and } x = 3} \).
\end{flushleft}

