
\subsection*{2.7 Remainder Arithmetic}
Remainder arithmetic, also known as modular arithmetic, deals with integers under division by a fixed number (the modulus). It is commonly used in number theory and cryptography.

\textbf{Key Concepts:}
\begin{itemize}
    \item \textbf{Modulo Operation:} The modulo operation finds the remainder when one integer is divided by another. For example, $11 \equiv 2 \,(\text{mod } 3)$, because 11 gives the remainder 2 when $8$ is divided by $3$ ($\frac{11}{3} = 3 + \frac{2}{3}$). \\Formally, $a \equiv r \,(\text{mod } m)$ if $a = qm + r$, where $q$ is the quotient and $0 \leq r < m$.

    \item \textbf{Congruence Modulo:} Two integers $a$ and $b$ are congruent modulo $m$, written as $a \equiv b \,(\text{mod } m)$, if their difference is divisible by $m$. Formally:
    \[
    a \equiv b \,(\text{mod } m) \iff m \,|\, (a - b).
    \]
    
    \item \textbf{Multiplication Table Modulo $m$:} A table showing the products of integers under modulo $m$, useful for understanding modular arithmetic operations.
\end{itemize}

\begin{flushleft}
\textbf{Example 1: Verify if $17 \equiv 5 \,(\text{mod } 6)$.}

\vspace{0.5cm}
\textbf{Solution:}
\vspace{0.5cm}

Check if $17 - 5$ is divisible by 6:
\[
17 - 5 = 12.
\]
Since $12$ is divisible by $6$, we conclude:
\[
17 \equiv 5 \,(\text{mod } 6).
\]
\end{flushleft}

\begin{flushleft}
\textbf{Example 2: Compute $(7 + 10) \,(\text{mod } 5)$.}

\vspace{0.5cm}
\textbf{Solution:}
\vspace{0.5cm}

Add the numbers:
\[
7 + 10 = 17.
\]

Find the remainder when $17$ is divided by $5$:
\[
17 \div 5 = 3 \text{ remainder } 2.
\]

Therefore:
\[
(7 + 10) \,(\text{mod } 5) = 2.
\]
\end{flushleft}

\begin{flushleft}
\textbf{Example 3: Compute $(9 \cdot 8) \,(\text{mod } 7)$.}

\vspace{0.5cm}
\textbf{Solution:}
\vspace{0.5cm}

Multiply the numbers:
\[
9 \cdot 8 = 72.
\]

Find the remainder when $72$ is divided by $7$:
\[
72 \div 7 = 10 \text{ remainder } 2.
\]

Therefore:
\[
(9 \cdot 8) \,(\text{mod } 7) = 2.
\]
\end{flushleft}

\begin{flushleft}
\textbf{Example 4: Compute $3^4 \,(\text{mod } 5)$.}

\vspace{0.5cm}
\textbf{Solution:}
\vspace{0.5cm}

Calculate the power:
\[
3^4 = 81.
\]

Find the remainder when $81$ is divided by $5$:
\[
81 \div 5 = 16 \text{ remainder } 1.
\]

Therefore:
\[
3^4 \,(\text{mod } 5) = 1.
\]
\end{flushleft}

\begin{flushleft}
\textbf{Example 5: Multiplication Table Modulo $4$.}

\vspace{0.5cm}
\textbf{Solution:}
\vspace{0.5cm}

Construct the table for integers $0, 1, 2, 3$ modulo $4$:
\[
\begin{array}{c|cccc}
\cdot \,(\text{mod } 4) & 0 & 1 & 2 & 3 \\\\ \hline
0 & 0 & 0 & 0 & 0 \\\\
1 & 0 & 1 & 2 & 3 \\\\
2 & 0 & 2 & 0 & 2 \\\\
3 & 0 & 3 & 2 & 1 \\\\
\end{array}
\]

This table shows the results of multiplication under modulo $4$.
\end{flushleft}
