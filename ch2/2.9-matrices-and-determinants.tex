
\subsection*{2.9 Matrices and Determinants}
Matrices are rectangular arrays of numbers arranged in rows and columns, used to represent and solve systems of linear equations and perform transformations in geometry.

\textbf{Key Concepts:}
\begin{itemize}
    \item \textbf{Matrix Definition:} A matrix is an array of numbers, symbols, or expressions arranged in rows and columns. For example:
    \[
    A = \begin{bmatrix} 1 & 2 & 3 \\\\ 4 & 5 & 6 \\\\ 7 & 8 & 9 \end{bmatrix}
    \]
    \item \textbf{Size of a Matrix:} The size of a matrix is defined as the number of rows $\times$ the number of columns. For example, the above matrix $A$ is a $3 \times 3$ matrix.
    \item \textbf{Basic Operations with Matrices:}
    \begin{itemize}
        \item \textbf{Addition and Subtraction:} Matrices of the same size are added or subtracted element by element.
        \item \textbf{Multiplication:} To multiply two matrices, the number of columns in the first matrix must equal the number of rows in the second matrix. \\
        The element in the $i$th row and $j$th column of the product matrix is the sum of the products of the corresponding elements of the $i$th row of the first matrix and the $j$th column of the second matrix. \\
        \[
\begin{bmatrix} a_1 & b_1 \\\\ c_1 & d_1 \end{bmatrix} \cdot \begin{bmatrix} a_2 & b_2 \\\\ c_2 & d_2 \end{bmatrix} = \begin{bmatrix} a_1a_2 + b_1c_2 & a_1b_2 + b_1d_2 \\\\ c_1a_2 + d_1c_2 & c_1b_2 + d_1d_2 \end{bmatrix}
\]

    \end{itemize}
    \item \textbf{Determinant:} The determinant is a scalar value associated with a square matrix, used to determine whether the matrix is invertible. For a $2 \times 2$ matrix:
    \[
    \text{If } A = \begin{bmatrix} a & b \\\\ c & d \end{bmatrix}, \quad \text{then } \det(A) = ad - bc.
    \]
    \item \textbf{Inverse Matrix:} The inverse of a square matrix $A$, denoted $A^{-1}$, exists if and only if $\det(A) \neq 0$. For a $2 \times 2$ matrix:
    \[
    A^{-1} = \frac{1}{\det(A)} \begin{bmatrix} d & -b \\\\ -c & a \end{bmatrix}.
    \]
\end{itemize}

\begin{flushleft}
\textbf{Example 1: Add the matrices $A = \begin{bmatrix} 1 & 2 \\\\ 3 & 4 \end{bmatrix}$ and $B = \begin{bmatrix} 5 & 6 \\\\ 7 & 8 \end{bmatrix}$.}

\vspace{0.5cm}
\textbf{Solution:}
\vspace{0.5cm}

Add corresponding elements:
\[
A + B = \begin{bmatrix} 1+5 & 2+6 \\\\ 3+7 & 4+8 \end{bmatrix} = \begin{bmatrix} 6 & 8 \\\\ 10 & 12 \end{bmatrix}.
\]

Therefore, $A + B = \begin{bmatrix} 6 & 8 \\\\ 10 & 12 \end{bmatrix}$.
\end{flushleft}

\begin{flushleft}
\textbf{Example 2: Find the determinant of $A = \begin{bmatrix} 3 & 4 \\\\ 2 & 1 \end{bmatrix}$.}

\vspace{0.5cm}
\textbf{Solution:}
\vspace{0.5cm}

Use the formula for the determinant of a $2 \times 2$ matrix:
\[
\det(A) = (3)(1) - (4)(2) = 3 - 8 = -5.
\]

Therefore, $\det(A) = -5$.
\end{flushleft}

\begin{flushleft}
\textbf{Example 3: Multiply the matrices $A = \begin{bmatrix} 1 & 2 \\\\ 3 & 4 \end{bmatrix}$ and $B = \begin{bmatrix} 2 & 0 \\\\ 1 & 3 \end{bmatrix}$.}

\vspace{0.5cm}
\textbf{Solution:}
\vspace{0.5cm}

Use the rule for matrix multiplication:
\[
A \cdot B = \begin{bmatrix} (1)(2)+(2)(1) & (1)(0)+(2)(3) \\\\ (3)(2)+(4)(1) & (3)(0)+(4)(3) \end{bmatrix} = \begin{bmatrix} 4 & 6 \\\\ 10 & 12 \end{bmatrix}.
\]

Therefore, $A \cdot B = \begin{bmatrix} 4 & 6 \\\\ 10 & 12 \end{bmatrix}$.
\end{flushleft}

\begin{flushleft}
\textbf{Example 4: Find the inverse of $A = \begin{bmatrix} 2 & 3 \\\\ 1 & 4 \end{bmatrix}$.}

\vspace{0.5cm}
\textbf{Solution:}
\vspace{0.5cm}

Step 1: Calculate the determinant:
\[
\det(A) = (2)(4) - (3)(1) = 8 - 3 = 5.
\]

Step 2: Use the formula for the inverse of a $2 \times 2$ matrix:
\[
A^{-1} = \frac{1}{\det(A)} \begin{bmatrix} d & -b \\\\ -c & a \end{bmatrix} = \frac{1}{5} \begin{bmatrix} 4 & -3 \\\\ -1 & 2 \end{bmatrix}.
\]

Therefore:
\[
A^{-1} = \begin{bmatrix} \frac{4}{5} & -\frac{3}{5} \\\\ -\frac{1}{5} & \frac{2}{5} \end{bmatrix}.
\]
\end{flushleft}
