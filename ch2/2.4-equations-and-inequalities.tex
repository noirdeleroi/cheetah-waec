
\subsection*{2.4 Equations and Inequalities}
Equations and inequalities are fundamental tools in algebra used to express relationships between variables and solve problems. 

\textbf{Key Concepts:}
\begin{itemize}
    \item \textbf{Linear Equations:} Equations of the form $ax + b = 0$, where $a \neq 0$.
    
    \item \textbf{Functional Equations:} Equations involving functions where the goal is to determine the function form, e.g., $f(x+1) = 2f(x)$.
    \item \textbf{Inequalities:} Statements involving $<$, $>$, $\leq$, or $\geq$ that describe the range of solutions for a variable.
\end{itemize}


\begin{flushleft}
\textbf{Example 1: Mary has \$3.00 more than Ben but \$5.00 less than Jane. If Mary has \$x, how much do Jane and Ben have altogether?}

\vspace{0.5cm}
\textbf{Solution:}
\vspace{0.5cm}

Step 1: Define the amounts Jane and Ben have in terms of Mary:
\[
\text{Ben's amount: } x - 3, \quad \text{Jane's amount: } x + 5.
\]

Step 2: Add their amounts:
\[
\text{Total: } (x - 3) + x + (x + 5).
\]

Step 3: Simplify:
\[
\text{Total: } 3x + 2.
\]

Therefore, Jane and Ben have \$\(3x + 2\) altogether.
\end{flushleft}



\begin{flushleft}
\textbf{Example 2: Solve the linear equation $2x - 5 = 7$.}

\vspace{0.5cm}
\textbf{Solution:}
\vspace{0.5cm}

Step 1: Add 5 to both sides:
\[
2x - 5 + 5 = 7 + 5 \implies 2x = 12.
\]

Step 2: Divide by 2:
\[
x = \frac{12}{2} = 6.
\]

Therefore, the solution is $x = 6$.
\end{flushleft}



\begin{flushleft}
\textbf{Example 3: Solve the inequality $2x - 3 > 7$.}

\vspace{0.5cm}
\textbf{Solution:}
\vspace{0.5cm}

Step 1: Add 3 to both sides:
\[
2x - 3 + 3 > 7 + 3 \implies 2x > 10.
\]

Step 2: Divide by 2:
\[
x > \frac{10}{2} \implies x > 5.
\]

Therefore, the solution is $x > 5$.
\end{flushleft}
\begin{flushleft}
\textbf{Example 4: Linear Equation with Irrational Coefficients}

Solve:  
\[
\sqrt{50}x - \sqrt{2} = 0
\]

\textbf{Solution:} \vspace{0.2cm}

Step 1: Simplify \(\sqrt{50}\):  
\[
\sqrt{50} = \sqrt{25 \cdot 2} = 5\sqrt{2}
\]

Now the equation becomes:  
\[
5\sqrt{2}x - \sqrt{2} = 0
\]

Step 2: Add \(\sqrt{2}\) to both sides:  
\[
5\sqrt{2}x = \sqrt{2}
\]

Step 3: Divide both sides by \(5\sqrt{2}\):  
\[
x = \frac{\sqrt{2}}{5\sqrt{2}} = \frac{1}{5}
\]

Therefore, the solution is \( \boxed{x = \frac{1}{5}} \).
\end{flushleft}


\begin{flushleft}
\textbf{Example 5: Solve the functional equation $f(x+1) = 2f(x)$, given $f(1) = 3$.}

\vspace{0.5cm}
\textbf{Solution:}
\vspace{0.5cm}

Step 1: Write the values of $f(x)$ using the functional equation:
\[
f(x+1) = 2f(x).
\]

Step 2: Find $f(2)$ using $f(1) = 3$:
\[
f(2) = 2f(1) = 2 \cdot 3 = 6.
\]

Step 3: Find $f(3)$ using $f(2) = 6$:
\[
f(3) = 2f(2) = 2 \cdot 6 = 12.
\]

Step 4: Generalize the pattern:
\[
f(x+1) = 2f(x) \implies f(x) = 3 \cdot 2^{x-1}.
\]

Therefore, the solution to the functional equation is:
\[
f(x) = 3 \cdot 2^{x-1}.
\]
\end{flushleft}
