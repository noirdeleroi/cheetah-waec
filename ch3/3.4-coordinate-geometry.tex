
\subsection*{3.4 Coordinate Geometry}
Coordinate geometry involves the study of geometric figures using the coordinate plane and algebraic techniques. It is fundamental for solving problems involving points, lines, and circles.

\textbf{Key Concepts:}
\begin{itemize}
	\item \textbf{Slope (Gradient) of a Line:} The slope of a line measures its steepness and is calculated as:
	\[
	m = \frac{y_2 - y_1}{x_2 - x_1},
	\]
	where $(x_1, y_1)$ and $(x_2, y_2)$ are two points on the line.
	\item \textbf{Equation of a Line:}
	\begin{itemize}
		\item \textbf{Slope-Intercept Form:} $y = mx + c$, where $m$ is the slope and $c$ is the $y$-intercept.
		\item \textbf{Point-Slope Form:} $y - y_1 = m(x - x_1)$, where $m$ is the slope and $(x_1, y_1)$ is a point on the line.
		\item \textbf{Two-Intercept Form:} $\frac{x}{a} + \frac{y}{b} = 1$, where $a$ and $b$ are the $x$- and $y$-intercepts, respectively.
	\end{itemize}
	\item \textbf{Parallel and Perpendicular Lines:}
	\begin{itemize}
		\item Parallel lines have the same slope: $m_1 = m_2$.
		\item Perpendicular lines have slopes that multiply to $-1$: $m_1 \cdot m_2 = -1$.
	\end{itemize}
	\item \textbf{Midpoint of a Line Segment:} The midpoint of a line segment with endpoints $(x_1, y_1)$ and $(x_2, y_2)$ is:
	\[
	M = \left( \frac{x_1 + x_2}{2}, \frac{y_1 + y_2}{2} \right).
	\]
	\item \textbf{Distance Between Two Points:} The distance between two points $(x_1, y_1)$ and $(x_2, y_2)$ is:
	\[
	d = \sqrt{(x_2 - x_1)^2 + (y_2 - y_1)^2}.
	\]
	\item \textbf{Equation of a Circle:} The equation of a circle with center $(h, k)$ and radius $r$ is:
	\[
	(x - h)^2 + (y - k)^2 = r^2.
	\]
	\item \textbf{Finding the Equation of a Line Parallel to a Given Line:} A line parallel to $y = mx + c$ has the same slope $m$ but a different $y$-intercept.
	\item \textbf{Finding the Equation of a Line Perpendicular to a Given Line:} A line perpendicular to $y = mx + c$ has a slope of $-\frac{1}{m}$.
\end{itemize}

\textbf{Examples:}

\begin{flushleft}
	\textbf{Example 1: Find the slope of the line passing through the points $(2, 3)$ and $(5, 7)$.}
	
	\vspace{0.5cm}
	\textbf{Solution:}
	\vspace{0.5cm}
	
	Use the slope formula:
	\[
	m = \frac{y_2 - y_1}{x_2 - x_1} = \frac{7 - 3}{5 - 2} = \frac{4}{3}.
	\]
	
	Therefore, the slope is $\frac{4}{3}$.
\end{flushleft}

\begin{flushleft}
	\textbf{Example 2: Write the equation of the line with slope $2$ passing through the point $(3, 4)$.}
	
	\vspace{0.5cm}
	\textbf{Solution:}
	\vspace{0.5cm}
	
	Use the point-slope form:
	\[
	y - y_1 = m(x - x_1).
	\]
	
	Substitute $m = 2$, $x_1 = 3$, and $y_1 = 4$:
	\[
	y - 4 = 2(x - 3).
	\]
	
	Simplify:
	\[
	y = 2x - 6 + 4 = 2x - 2.
	\]
	
	Therefore, the equation is $y = 2x - 2$.
\end{flushleft}

\begin{flushleft}
	\textbf{Example 3: Find the distance between the points $(1, 2)$ and $(4, 6)$.}
	
	\vspace{0.5cm}
	\textbf{Solution:}
	\vspace{0.5cm}
	
	Use the distance formula:
	\[
	d = \sqrt{(x_2 - x_1)^2 + (y_2 - y_1)^2}.
	\]
	
	Substitute the values:
	\[
	d = \sqrt{(4 - 1)^2 + (6 - 2)^2} = \sqrt{3^2 + 4^2} = \sqrt{9 + 16} = \sqrt{25} = 5.
	\]
	
	Therefore, the distance is $5$ units.
\end{flushleft}

\begin{flushleft}
	\textbf{Example 4: Find the equation of a circle with center $(2, -3)$ and radius $5$.}
	
	\vspace{0.5cm}
	\textbf{Solution:}
	\vspace{0.5cm}
	
	Use the equation of a circle:
	\[
	(x - h)^2 + (y - k)^2 = r^2.
	\]
	
	Substitute $h = 2$, $k = -3$, and $r = 5$:
	\[
	(x - 2)^2 + (y + 3)^2 = 25.
	\]
	
	Therefore, the equation of the circle is:
	\[
	(x - 2)^2 + (y + 3)^2 = 25.
	\]
\end{flushleft}

\begin{flushleft}
	\textbf{Example 5: Find the equation of a line perpendicular to $y = \frac{1}{2}x + 3$ that passes through the point $(4, 2)$.}
	
	\vspace{0.5cm}
	\textbf{Solution:}
	\vspace{0.5cm}
	
	Step 1: Find the slope of the perpendicular line:
	\[
	m_{\text{perpendicular}} = -\frac{1}{m_{\text{original}}} = -\frac{1}{\frac{1}{2}} = -2.
	\]
	
	Step 2: Use the point-slope form:
	\[
	y - y_1 = m(x - x_1).
	\]
	
	Substitute $m = -2$, $x_1 = 4$, and $y_1 = 2$:
	\[
	y - 2 = -2(x - 4).
	\]
	
	Simplify:
	\[
	y = -2x + 8 + 2 = -2x + 10.
	\]
	
	Therefore, the equation of the line is $y = -2x + 10$.
\end{flushleft}

\begin{flushleft}
	\textbf{Example 6: The $x$- and $y$-intercepts of a straight line are $-\frac{3}{4}$ and $\frac{2}{7}$, respectively. Find the equation of the line.}
	
	\vspace{0.5cm}
	\textbf{Solution:}
	\vspace{0.5cm}
	
	Step 1: Use the two-intercept form of a line:
	\[
	\frac{x}{a} + \frac{y}{b} = 1,
	\]
	where $a$ is the $x$-intercept and $b$ is the $y$-intercept.
	
	Step 2: Substitute $a = -\frac{3}{4}$ and $b = \frac{2}{7}$:
	\[
	\frac{x}{-\frac{3}{4}} + \frac{y}{\frac{2}{7}} = 1.
	\]
	
	Step 3: Simplify the fractions:
	\[
	-\frac{4x}{3} + \frac{7y}{2} = 1.
	\]
	
	Step 4: Eliminate the fractions by multiplying through by the least common denominator (LCD), which is $6$:
	\[
	6 \times \left(-\frac{4x}{3}\right) + 6 \times \left(\frac{7y}{2}\right) = 6 \cdot 1.
	\]
	
	Simplify:
	\[
	-8x + 21y = 6.
	\]
	
	Therefore, the equation of the line is:
	\[
	-8x + 21y = 6.
	\]
	
	Alternatively, in point-slope form:
	Using the slope $m = \frac{\Delta y}{\Delta x} = \frac{\frac{2}{7} - 0}{0 - \left(-\frac{3}{4}\right)} = \frac{\frac{2}{7}}{\frac{3}{4}} = \frac{8}{21}$, and passing through the point $(0, \frac{2}{7})$, the equation becomes:
	\[
	y - \frac{2}{7} = \frac{8}{21}(x - 0).
	\]
	
	Therefore, the gradient-point form is:
	\[
	y = \frac{8}{21}x + \frac{2}{7}.
	\]
\end{flushleft}
