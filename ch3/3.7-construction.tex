

\subsection*{3.7 Construction}
Construction in geometry involves creating geometric shapes, angles, and lines using a ruler and a compass. It is a fundamental skill for solving problems that require precision and accuracy.

\textbf{Key Concepts:}
\begin{itemize}
	\item \textbf{Basic Tools for Construction:}
	\begin{itemize}
		\item \textbf{Ruler:} Used to draw straight lines and measure distances.
		\item \textbf{Compass:} Used to draw circles and arcs and to measure distances that can be transferred onto diagrams.
	\end{itemize}
	\item \textbf{Constructing a Perpendicular Bisector:}
	The perpendicular bisector of a line segment divides it into two equal parts and forms a $90^\circ$ angle with the line segment.
	\item \textbf{Bisecting an Angle:}
	The angle bisector divides an angle into two equal parts.
	\item \textbf{Constructing a Triangle:}
	A triangle can be constructed given the following:
	\begin{itemize}
		\item Three sides (SSS).
		\item Two sides and the included angle (SAS).
		\item Two angles and one side (ASA).
	\end{itemize}
	\item \textbf{Constructing Loci:}
	Using a compass and ruler, loci of points can be constructed based on specific geometric conditions, such as points equidistant from a fixed point or line.
\end{itemize}

\textbf{Examples:}

\begin{flushleft}
	\textbf{Example 1: Construct the perpendicular bisector of a line segment $AB$.}
	
	\vspace{0.5cm}
	\textbf{Solution:}
	\vspace{0.5cm}
	
	1. Place the compass at $A$ and draw an arc above and below the line segment. \\
	2. Without changing the compass width, place the compass at $B$ and draw arcs above and below the line segment, intersecting the first arcs. \\
	3. Use the ruler to draw a straight line through the points of intersection. \\
	The resulting line is the perpendicular bisector of $AB$.
\end{flushleft}

\begin{flushleft}
	\textbf{Example 2: Bisect a $60^\circ$ angle.}
	
	\vspace{0.5cm}
	\textbf{Solution:}
	\vspace{0.5cm}
	
	1. Draw a $60^\circ$ angle using a protractor. Label the vertex $O$. \\
	2. Place the compass at $O$ and draw an arc intersecting both arms of the angle. Label the points of intersection $A$ and $B$. \\
	3. Place the compass at $A$ and $B$ and draw two arcs that intersect each other inside the angle. Label the point of intersection $P$. \\
	4. Draw a straight line from $O$ to $P$. \\
	The line $OP$ bisects the $60^\circ$ angle into two $30^\circ$ angles.
\end{flushleft}

\begin{flushleft}
	\textbf{Example 3: Construct a triangle given sides $AB = 5 \, \text{cm}$, $AC = 4 \, \text{cm}$, and $BC = 6 \, \text{cm}$.}
	
	\vspace{0.5cm}
	\textbf{Solution:}
	\vspace{0.5cm}
	
	1. Draw a base line $BC = 6 \, \text{cm}$. \\
	2. Place the compass at $B$ and draw an arc of radius $5 \, \text{cm}$. \\
	3. Place the compass at $C$ and draw an arc of radius $4 \, \text{cm}$. \\
	4. Label the point of intersection of the arcs as $A$. \\
	5. Join $A$ to $B$ and $A$ to $C$ with straight lines. \\
	The resulting triangle $ABC$ satisfies the given dimensions.
\end{flushleft}

\begin{flushleft}
	\textbf{Example 4: Construct the locus of points $3 \, \text{cm}$ away from a given line $L$.}
	
	\vspace{0.5cm}
	\textbf{Solution:}
	\vspace{0.5cm}
	
	1. Place the compass at a point on $L$ and draw arcs of radius $3 \, \text{cm}$ on both sides of the line. \\
	2. Repeat the process at multiple points along $L$. \\
	3. Use the ruler to draw two parallel lines through the arc intersections. \\
	The resulting lines are the locus of points $3 \, \text{cm}$ away from $L$.
\end{flushleft}

