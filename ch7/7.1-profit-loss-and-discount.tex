
\subsection*{7.1 Profit, Loss, and Discount}
Commercial mathematics is essential in business transactions involving cost, revenue, and profit calculations.

\textbf{Key Concepts:}
\begin{itemize}
	\item \textbf{Cost Price (CP):} The original price at which an item is purchased.
	\item \textbf{Selling Price (SP):} The price at which an item is sold.
	\item \textbf{Profit:} The gain obtained when an item is sold for more than its cost price.
	\[
	\text{Profit} = \text{SP} - \text{CP}
	\]
	\item \textbf{Loss:} The amount lost when an item is sold for less than its cost price.
	\[
	\text{Loss} = \text{CP} - \text{SP}
	\]
	\item \textbf{Profit Percentage:}
	\[
	\text{Profit \%} = \left( \frac{\text{Profit}}{\text{CP}} \right) \times 100
	\]
	\item \textbf{Loss Percentage:}
	\[
	\text{Loss \%} = \left( \frac{\text{Loss}}{\text{CP}} \right) \times 100
	\]
	\item \textbf{Discount:} A reduction in the marked price of an item.
	\[
	\text{Discount} = \text{Marked Price} - \text{Selling Price}
	\]
	\item \textbf{Discount Percentage:}
	\[
	\text{Discount \%} = \left( \frac{\text{Discount}}{\text{Marked Price}} \right) \times 100
	\]
	\item \textbf{Markup:} The percentage increase on the cost price before selling.
	\[
	\text{Selling Price} = \text{Cost Price} + \left( \frac{\text{Markup \%} \times \text{CP}}{100} \right)
	\]
\end{itemize}

\textbf{Examples:}

\begin{flushleft}
	\textbf{Example 1: Finding Profit Percentage}
	
	A trader buys a shirt for \$40 and sells it for \$50. Find the profit percentage.
	
	\textbf{Solution:}
	
	Step 1: Compute the profit.
	\[
	\text{Profit} = 50 - 40 = 10.
	\]
	
	Step 2: Compute the profit percentage.
	\[
	\text{Profit \%} = \left( \frac{10}{40} \right) \times 100 = 25\%.
	\]
	
	Thus, the profit percentage is **25\%**.
\end{flushleft}

\begin{flushleft}
	\textbf{Example 2: Finding Discount Percentage}
	
	A store sells a bicycle originally marked at \$200 for \$170 after applying a discount. Find the discount percentage.
	
	\textbf{Solution:}
	
	Step 1: Compute the discount.
	\[
	\text{Discount} = 200 - 170 = 30.
	\]
	
	Step 2: Compute the discount percentage.
	\[
	\text{Discount \%} = \left( \frac{30}{200} \right) \times 100 = 15\%.
	\]
	
	Thus, the discount percentage is **15\%**.
\end{flushleft}

\begin{flushleft}
	\textbf{Example 3: Calculating Actual Percentage Profit}
	
	A television set was marked for sale at N760.00 in order to make a profit of 20\%. The television set was actually sold at a discount of 5\%. Calculate, correct to 2 significant figures, the actual percentage profit.
	
	\textbf{Solution:}
	
	Step 1: Find the cost price (CP).
	
	Since the marked price includes a 20\% profit on the cost price, we set up the equation:
	\[
	\text{Marked Price} = \text{CP} + 20\% \times \text{CP} = 1.2 \times \text{CP}
	\]
	Given the marked price is N760:
	\[
	\text{CP} = \frac{760}{1.2} = 633.33
	\]
	
	Step 2: Find the actual selling price (SP).
	
	Since a 5\% discount is applied:
	\[
	\text{SP} = \text{Marked Price} - 5\% \times \text{Marked Price}
	\]
	\[
	\text{SP} = 760 - (0.05 \times 760) = 760 - 38 = 722
	\]
	
	Step 3: Calculate the actual profit.
	
	\[
	\text{Profit} = \text{SP} - \text{CP} = 722 - 633.33 = 88.67
	\]
	
	Step 4: Find the actual profit percentage.
	
	\[
	\text{Profit \%} = \left(\frac{\text{Profit}}{\text{CP}}\right) \times 100 = \left(\frac{88.67}{633.33}\right) \times 100
	\]
	
	\[
	= 14.0\%
	\]
	
	Thus, the actual percentage profit is **14\%** (correct to 2 significant figures).
\end{flushleft}
