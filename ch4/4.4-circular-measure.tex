
\subsection*{4.4 Circular Measure}
Circular measure involves working with angles in both degrees and radians, as well as calculating arc lengths and sector areas.

\textbf{Key Concepts:}

\begin{itemize}
	\item \textbf{Radians and Degrees:} 
	- A radian is an alternative unit for measuring angles, where one full revolution (360°) equals $2\pi$ radians.
	- The conversion formulas between degrees and radians are:
	\[
	1^\circ = \frac{\pi}{180} \text{ radians}, \quad 1 \text{ radian} = \frac{180}{\pi}^\circ.
	\]
	
	\item \textbf{Arc Length:} 
	- The length of an arc of a circle is given by:
	\[
	l = r\theta,
	\]
	where $r$ is the radius and $\theta$ is the central angle in radians.
	
	\item \textbf{Sector Area:}
	- The area of a sector of a circle is given by:
	\[
	A = \frac{1}{2} r^2 \theta,
	\]
	where $r$ is the radius and $\theta$ is the central angle in radians.
	
	\item \textbf{Applications:} 
	- Circular measure is used in physics, engineering, and navigation for measuring distances along circular paths.
\end{itemize}

\textbf{Examples:}

\begin{flushleft}
	\textbf{Example 1: Convert $135^\circ$ to radians.}
	
	\vspace{0.5cm}
	\textbf{Solution:}
	\vspace{0.5cm}
	
	Using the conversion formula:
	\[
	\theta = 135^\circ \times \frac{\pi}{180}.
	\]
	
	\[
	\theta = \frac{135\pi}{180}.
	\]
	
	\[
	\theta = \frac{3\pi}{4} \text{ radians}.
	\]
	
	Thus, $135^\circ = \frac{3\pi}{4}$ radians.
\end{flushleft}

\begin{flushleft}
	\textbf{Example 2: Convert $2.5$ radians to degrees.}
	
	\vspace{0.5cm}
	\textbf{Solution:}
	\vspace{0.5cm}
	
	Using the conversion formula:
	\[
	\theta = 2.5 \times \frac{180}{\pi}.
	\]
	
	\[
	\theta = \frac{450}{\pi} \approx 143.24^\circ.
	\]
	
	Thus, $2.5$ radians $\approx 143.2^\circ$.
\end{flushleft}

\begin{flushleft}
	\textbf{Example 3: Find the length of an arc in a circle of radius $10$ cm subtended by a central angle of $1.2$ radians.}
	
	\vspace{0.5cm}
	\textbf{Solution:}
	\vspace{0.5cm}
	
	Using the arc length formula:
	\[
	l = r\theta.
	\]
	
	Substituting values:
	\[
	l = 10 \times 1.2 = 12 \text{ cm}.
	\]
	
	Thus, the arc length is $12$ cm.
\end{flushleft}

\begin{flushleft}
	\textbf{Example 4: Find the area of a sector with radius $8$ cm and central angle $2$ radians.}
	
	\vspace{0.5cm}
	\textbf{Solution:}
	\vspace{0.5cm}
	
	Using the sector area formula:
	\[
	A = \frac{1}{2} r^2 \theta.
	\]
	
	Substituting values:
	\[
	A = \frac{1}{2} \times 8^2 \times 2.
	\]
	
	\[
	A = \frac{1}{2} \times 64 \times 2 = 64 \text{ cm}^2.
	\]
	
	Thus, the area of the sector is $64$ cm².
\end{flushleft}
