
\section*{Chapter 6: Probability}

\subsection*{6.1 Simple Probability}
Probability is the measure of how likely an event is to occur. It is expressed as a number between 0 (impossible) and 1 (certain). 

\textbf{Key Concepts:}
\begin{itemize}
	\item \textbf{Experiment:} A process that leads to an outcome.
	\item \textbf{Sample Space (S):} The set of all possible outcomes of an experiment.
	\item \textbf{Event (E):} A subset of the sample space representing specific outcomes.
	\item \textbf{Outcome:} A single possible result of an experiment. 
	\begin{itemize}
		\item Example: Rolling a die can result in outcomes $1,2,3,4,5,$ or $6$.
		\item Example: Flipping a coin has two possible outcomes: **Heads** or **Tails**.
	\end{itemize}
	
	\item \textbf{Outcome Space (Sample Space, S):} The set of all possible outcomes of an experiment.
	\begin{itemize}
		\item Example: The sample space for rolling a six-sided die is:
		\[
		S = \{1,2,3,4,5,6\}.
		\]
		\item Example: The sample space for flipping two coins is:
		\[
		S = \{\text{HH, HT, TH, TT}\}.
		\]
	\end{itemize}
	
	\item \textbf{Probability of an Event:} The probability of an event occurring is given by:
	\[
	P(E) = \frac{\text{Number of favorable outcomes}}{\text{Total number of possible outcomes}}.
	\]
	\item \textbf{Complement of an Event:} The probability of an event not occurring is:
	\[
	P(E') = 1 - P(E).
	\]
	
\end{itemize}

\textbf{Examples:}

\begin{flushleft}
	\textbf{Example 1: Finding the Probability of Rolling a Die}
	
	\textbf{Solution:}
	
	Step 1: Identify the sample space. \\
	The sample space for rolling a die is:
	\[
	S = \{1,2,3,4,5,6\}.
	\]
	
	Step 2: Find the probability of rolling a 4.
	\[
	P(4) = \frac{1}{6}.
	\]
	
	Thus, the probability of rolling a 4 is **$\frac{1}{6}$**.
\end{flushleft}

\begin{flushleft}
	\textbf{Example 2: Probability of Drawing a Red Card from a Deck of 52 Cards}
	
	\textbf{Solution:}
	
	Step 1: Identify the total outcomes. \\
	A standard deck has 52 cards, and half (26) are red.
	
	Step 2: Compute the probability.
	\[
	P(\text{Red Card}) = \frac{26}{52} = \frac{1}{2}.
	\]
	
	Thus, the probability of drawing a red card is **$\frac{1}{2}$**.
\end{flushleft}
