\documentclass{article}
\usepackage{graphicx} % Required for inserting images

\title{1.2}
\author{Dmitrii Korshunov}
\date{April 2025}

\begin{document}


\subsection*{1.2 Indices}

Indices (or exponents) are used to represent repeated multiplication of a number by itself. Understanding the laws of indices is essential for simplifying algebraic expressions involving powers.

\textbf{Key Concepts:}

\begin{itemize}
    \item \textbf{Product Rule:} When multiplying powers with the same base, add the exponents:
    \[
    a^m \cdot a^n = a^{m+n}
    \]

    \item \textbf{Quotient Rule:} When dividing powers with the same base, subtract the exponents:
    \[
    \frac{a^m}{a^n} = a^{m-n} \quad (a \neq 0)
    \]

    \item \textbf{Power of a Power:} Multiply the exponents:
    \[
    (a^m)^n = a^{mn}
    \]

    \item \textbf{Zero Exponent:} Any non-zero base raised to the power of 0 is 1:
    \[
    a^0 = 1 \quad (a \neq 0)
    \]

    \item \textbf{Negative Exponent:} A negative exponent indicates a reciprocal:
    \[
    a^{-n} = \frac{1}{a^n}
    \]

    \item \textbf{Fractional Exponent:} A fractional exponent represents a root (find more in Chapter 1.3):
    \[
    a^{\frac{m}{n}} = \sqrt[n]{a^m}
    \]
\end{itemize}

\begin{flushleft}
\textbf{Example 1: Applying the Laws of Indices}

Simplify:  
\[
\frac{3^4 \cdot 3^{-2}}{3}
\]

\textbf{Solution:} \vspace{0.2cm}

Step 1: Use the product rule on the numerator:  
\[
3^4 \cdot 3^{-2} = 3^{4 + (-2)} = 3^2
\]

Step 2: Divide by \(3 = 3^1\):  
\[
\frac{3^2}{3^1} = 3^{2-1} = 3^1 = 3
\]

Therefore, the answer is:  
\[
3
\]
\end{flushleft}

\begin{flushleft}
\textbf{Example 2: Zero and Negative Exponents}

Simplify:  
\[
2^0 + 5^{-2}
\]

\textbf{Solution:} \vspace{0.2cm}

Step 1: Apply the zero exponent rule:  
\[
2^0 = 1
\]

Step 2: Apply the negative exponent rule:  
\[
5^{-2} = \frac{1}{5^2} = \frac{1}{25}
\]

Step 3: Add the results:  
\[
1 + \frac{1}{25} = \frac{26}{25}
\]
\end{flushleft}

\begin{flushleft}
\textbf{Example 3: Fractional Exponents}

Simplify:  
\[
16^{\frac{3}{4}}
\]

\textbf{Solution:} \vspace{0.2cm}

Step 1: Use the fractional exponent rule:  
\[
16^{\frac{3}{4}} = \left( \sqrt[4]{16} \right)^3
\]

Step 2: Simplify:  
\[
\sqrt[4]{16} = 2 \quad \Rightarrow \quad 2^3 = 8
\]

Therefore, the answer is:  
\[
8
\]
\end{flushleft}


\end{document}
