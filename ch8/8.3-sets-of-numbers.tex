
\subsection*{8.3 Sets of Numbers}

Numbers can be categorized into different sets based on their properties. These sets form the foundation of number theory and algebra.

\textbf{Key Concepts:}
\begin{itemize}
	\item \textbf{Natural Numbers ($\mathbb{N}$):} The set of positive counting numbers:
	\[
	\mathbb{N} = \{1, 2, 3, 4, 5, \dots\}
	\]
	Some definitions include 0 as a natural number: $\mathbb{N}_0 = \{0, 1, 2, 3, \dots\}$.
	
	\item \textbf{Whole Numbers ($\mathbb{W}$):} The set of natural numbers including zero:
	\[
	\mathbb{W} = \{0, 1, 2, 3, \dots\}
	\]
	
	\item \textbf{Integers ($\mathbb{Z}$):} The set of whole numbers and their negative counterparts:
	\[
	\mathbb{Z} = \{\dots, -3, -2, -1, 0, 1, 2, 3, \dots\}
	\]
	
	\item \textbf{Rational Numbers ($\mathbb{Q}$):} Numbers that can be expressed as a fraction $\frac{a}{b}$, where $a, b \in \mathbb{Z}$ and $b \neq 0$.
	\[
	\mathbb{Q} = \left\{\frac{a}{b} \mid a, b \in \mathbb{Z}, b \neq 0 \right\}
	\]
	Examples: $\frac{1}{2}, -3, 0.75, 5$.
	
	\item \textbf{Irrational Numbers:} Numbers that cannot be written as fractions, having non-repeating, non-terminating decimals.
	\[
	\sqrt{2}, \pi, e
	\]
	
	\item \textbf{Real Numbers ($\mathbb{R}$):} The set of all rational and irrational numbers.
	\[
	\mathbb{R} = \mathbb{Q} \cup \text{Irrational Numbers}
	\]
	
	\item \textbf{Complex Numbers ($\mathbb{C}$):} The set of numbers in the form:
	\[
	a + bi, \quad \text{where } a, b \in \mathbb{R} \text{ and } i = \sqrt{-1}.
	\]
\end{itemize}

\textbf{Examples:}

\begin{flushleft}
	\textbf{Example 1: Classify the number $\frac{5}{2}$.}
	
	\textbf{Solution:}  
	$\frac{5}{2}$ is a fraction of two integers, so it belongs to **rational numbers ($\mathbb{Q}$)**. Since it is not a whole number, it is not an integer.
\end{flushleft}

\begin{flushleft}
	\textbf{Example 2: Determine whether $\sqrt{16}$ and $\sqrt{17}$ are rational.}
	
	\textbf{Solution:}  
	\[
	\sqrt{16} = 4, \quad \text{which is an integer and therefore rational.}
	\]
	\[
	\sqrt{17} \approx 4.123, \quad \text{which is a non-repeating, non-terminating decimal, so it is irrational.}
	\]
	Thus, $\sqrt{16}$ is **rational**, while $\sqrt{17}$ is **irrational**.
\end{flushleft}
